\documentclass{article}


\usepackage[superscript,biblabel]{cite}
% \usepackage[super,numbers]{natbib}
\bibliographystyle{naturemag}
% \bibliographystyle{plain}

\usepackage{booktabs}%


\begin{document}

\title{Population scale Ancestral Recombination Graphs with tskit 1.0}
\author{Author list to be filled in}
\maketitle

Ancestral recombination graphs (ARGs) capture the evolutionary
history of samples in a recombining species~\cite{wong2024general}
ARGs have been a central object in population genetics
for decades, but were limited in practical terms by the lack of
scalble inference methods, interchange formats and software infrastructure.
Recent breakthroughs in simulation and inference have changed that landscape
and there is now great interest in ARGs and their 
applications~\cite{brandt2024promise,lewanski2024era,nielsen2024inference}.
Tskit is the core ARG software library,
with and efficient interchange format an a rich suite of utilities,
and has been a key
catalyst in this ongoing ARG revolution. This paper marks the
release of tskit 1.0, which formalises long-term stability
guarantees, ensuring that it remains dependable infrastructure
for this quickly growing field.

At the heart of tskit is a the ``succinct tree sequence'' data model,
in which a set of nodes (ancestral genomes at particular times) and
edges (parent-child inheritance relationships spanning genomic intervals)
are recorded in a simple tabular format~\cite{kelleher2018efficient}.
This encoding leads to an efficient way of retrieving the (typically
small) differences between local trees along the genome, and
for efficient algorithms that leverage this property. The tskit data
model also incorporates pedigree and population information, as well
as supporting arbitrary metadata associated with different apects of the
data model. Provenance is also built into the data model, ensuring
that reproducibility.

The initial application of tskit was in simulation. Introduced initially as
part of the msprime simulator, the efficient data structures enabled performance
improvements of up to 6 orders of magnitude over existing
approaches~\cite{kelleher2016efficient}.
Similarly, the methods introduced for recording an ARG forwards in time
with periodic ``simplification'' led to up to 50X speedup over the standard approach
by reducing the need for simulating neutral
mutations~\cite{kelleher2018efficient}. 
It has also enabled
the simulation of ARGs under complex patterns of geographies and selection
which was not previously possible (vital ground truth for evaluating
inference methods). Simulation methods have continued to
refine and expand, with perhaps the most sophisticated to date being
a simulated whole-genome ARGs for nearly 1.5 million 
French-Canadians based on a large pedigree~\cite{andersontrocme2023genes}.
There is now rich and growing ecosystem of simulation tools built
directly around tskit (Fig 1).

Tskit has also had a significant influence on recent developments in
ARG inference. Building directly on tskit's data model and with
efficient incremental algorithms enabled by it, tsinfer was the
first ARG inference method to scale to hundreds of thousands of
samples~\cite{kelleher2019inferring}. 
Tskit is now the de-facto interchange standard among ARG inference
methods, with output support included in all recently published
methods. 
Most recently, the algorithms in tsinfer have been refined
and extended to infer an ARG for 2.48 million SARS-CoV-2 whole genomes,
illustrating both the scalability of the approach and the flexibility
in the underlying model in handling viral genomic data along
with humans~\cite{zhan2025pandemic}. The ARG requires 32MiB of storage and can be loaded
into memory in less than a second~\cite{zhan2025pandemic}.

Efficient storage and analysis of large datasets is a key goal of tskit,
leveraging the underlying data structure for major performance gains. For
example, single site statistics like Tajima's D can be computed up to 1000
times faster than is possible from genotype matrix and using much less memory,
as well as providing a "branch-based" dual~\cite{ralph2020efficiently}. 
Tskit has a large API with many
functions spanning statistical calculations to visualisation. The core is
written in C, with bindings in Python, Rust and R. Its vectorised, table-first
design makes it straightforward to expose zero-copy views into the underlying
arrays (for example through NumPy), supporting high-performance analysis
pipelines that avoid unnecessary memory duplication. This design efficient
cross-language interfaces, and helps ensure that downstream tools inherit
performance and correctness properties from a shared core.

Tskit is now foundational infrastructure for the field. Table 1
shows 56 published software tools that rely on tskit software,
spanning population genetic and statistical genetic inference,
through basic utilities, and a rapidly growing suite of inference
methods.
The 1.0 release marks the maturity of tskit as a shared community resource,
with long-term stability of the data model and APIs guaranteed.
This
stability, and the global community of contributors, ensures that tskit
can be built upon with confidence and is a suitable format for
long-term scientific value and will enable tools to interoperate through
a common exchange format.
In combination with an active ecosystem (spanning simulation,
inference, statistics, and visualisation) tskit 1.0 marks a transition from a
successful library to stable infrastructure for population-scale genealogical
analysis.


\begin{table}
\caption{The tskit software ecosystem. * data not available on OpenAlex}
\begin{scriptsize}
\begin{center}
\begin{tabular}{lllrr}
\toprule
name & interface & package & year & citations \\
\midrule
\multicolumn{5}{c}{\textsc{Simulation}} \\
\midrule
pyslim & Python & pypi, conda & 2025 & 1 \\
TwisstNTern & Python &  & 2025 & 0 \\
tstrait & Python & pypi, conda & 2024 & 8 \\
slendr & Python & CRAN & 2023 & 15 \\
cnasim & Python & pypi & 2023 & 10 \\
gridCoal & Python &  & 2022 & 5 \\
Geonomics & Python & pypi & 2021 & 20 \\
stdpopsim & Python & pypi, conda & 2020 & 233 \\
RADinitio & Python & pypi & 2020 & 49 \\
fwdpy11 & C, Python & pypi, conda & 2019 & 74 \\
SLiM & C &  & 2018 & 860 \\
msprime & C, Python & pypi, conda & 2016 & 806 \\

\midrule
\multicolumn{5}{c}{\textsc{ARG Inference}} \\
\midrule
SINGER & Python &  & 2025 & 12 \\
sticcs & Python &  & 2025 & 0 \\
POLEGON & Python &  & 2025 & 0 \\
Threads & Python & pypi & 2024 & 22 \\
espalier & Python & pypi & 2023 & 14 \\
sc2ts & Python & pypi & 2023 & 14 \\
tsdate & Python & pypi, conda & 2022 & 187 \\
ARGinfer & Python & pypi & 2022 & 37 \\
Relate & C &  & 2019 & 562 \\
tsinfer & C, Python & pypi, conda & 2019 & 373 \\
ARGneedle & Python & pypi & * &  \\

\midrule
\multicolumn{5}{c}{\textsc{Population Genetic Inference}} \\
\midrule
gaia & C &  & 2025 & 8 \\
mrpast & Python & pypi & 2025 & 0 \\
phlash & Python & pypi & 2025 & 0 \\
sparg & Python & pypi & 2025 & 0 \\
spacetrees & Python &  & 2024 & 18 \\
mapNN & Python &  & 2024 & 4 \\
gIMble & Python & conda & 2023 & 33 \\
disperseNN2 & Python & pypi & 2023 & 10 \\
dinf & Python & pypi & 2023 & 8 \\
GADMA & Python & pypi & 2022 & 23 \\
dnadna & Python & pypi, conda & 2022 & 8 \\
DISMaL & Python & pypi & 2021 & 7 \\
ReLERNN & Python & pypi & 2020 & 166 \\
momi2 & Python & pypi & 2019 & 170 \\

\midrule
\multicolumn{5}{c}{\textsc{Statistical Genetic Inference}} \\
\midrule
as-eGRM & Python &  & 2025 & 1 \\
arg-lmm & Python & pypi & 2025 & 0 \\
tslmm & Python &  &  &  \\
lgdm & Python & pypi & 2023 & 38 \\
egrm & Python & pip & 2022 & 44 \\

\midrule
\multicolumn{5}{c}{\textsc{Visualisation}} \\
\midrule
tsbrowse & Python & pypi & 2025 & 0 \\
ARGscape & Python & pypi & 2025 & 0 \\
tskit-arg-visualizer & Python & pypi & 2024 & 0 \\

\midrule
\multicolumn{5}{c}{\textsc{Data processing}} \\
\midrule
bio2zarr & python & pypi, conda & 2025 & 0 \\

\midrule
\multicolumn{5}{c}{\textsc{Analysis}} \\
\midrule
tscompare & Python & pypi & 2025 & 1 \\
twisst2 & Python &  & 2025 & 0 \\
tspop & Python & pypi & 2023 & 12 \\
\bottomrule
\end{tabular}
\end{center}
\end{scriptsize}
\end{table}


% Citations:

% - wong2024general
% - brandt2024promise
% - lewanski2024era
% - nielsen2024inference
% - kelleher2016efficient
% - kelleher2016efficient
% - kelleher2019inferring
% - ralph2020efficiently
% - andersontrocme2023genes
% - zhan2025pandemic


\section*{Acknowledgments}

\bibliography{paper}


\end{document}
