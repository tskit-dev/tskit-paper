\documentclass{article}


\usepackage[superscript,biblabel]{cite}
% \usepackage[super,numbers]{natbib}
\bibliographystyle{naturemag}
% \bibliographystyle{plain}

\usepackage{booktabs}%


\begin{document}

\title{Population scale Ancestral Recombination Graphs with tskit 1.0}
\author{Author list to be filled in}
\maketitle

% ARGs are now practical, and tskit is the infrastructure enabling their use.

\noindent
Ancestral recombination graphs (ARGs) capture the full genealogical history of
samples from a recombining species. 
Although ARGs have been a central
theoretical object in population genetics for decades, their practical use was
constrained by the lack of scalable inference methods, standard
interchange formats, and software infrastructure. Recent breakthroughs in
simulation and inference have substantially changed this landscape, 
leading to renewed interest in ARG-based analyses across population and statistical
genetics~\cite{brandt2024promise,lewanski2024era,nielsen2024inference}.
The tskit library has played a key enabling role in this shift
and has become foundational infrastructure for working with ARGs at scale. 
This paper marks the release of tskit 1.0, which formalises long-term stability
guarantees for its data model and APIs.

% The succinct tree sequence is a complete, lossless encoding of ARGs.
At the core of tskit is the succinct tree sequence data model, which 
defines a set of nodes (ancestral genomes at particular times) and 
edges (parent-child inheritance relationships spanning genomic intervals) 
in a simple tabular form\cite{kelleher2018efficient}.
This encoding provides a lossless representation of a general class of
ancestral recombination graphs, 
establishing a precise and machine-readable
definition of ARGs suitable for large-scale computation\cite{wong2024general}. 
The data model also incorporates site, mutation, population, and pedigree 
information and supports arbitrary
metadata associated with different components of the ARG. 
Provenance information is recorded natively, enhancing reproducibility 
and transparency.
Together, these features make tskit a semantically complete and interoperable
representation of ARGs that serves as a common foundation across diverse
analytical workflows.

% This encoding first proved itself by transforming population-genetic
% simulation.

Simulation is a fundamental tool in population genomics, and 
was the first domain in which the tskit data model demonstrated its impact.
Introduced initially as part of the msprime simulator,
the use of succinct tree sequences enabled performance improvements of 
several orders of magnitude
over previous coalescent simulation approaches\cite{kelleher2016efficient}. 
The same representation later
enabled efficient forward-time simulation of ARGs
and yielded substantial speedups by avoiding
explicit simulation of neutral mutations\cite{kelleher2018efficient}.
Because these forward-time and coalescent
simulators share a common underlying representation, their complementary
strengths can be combined within a single workflow. This has made it possible
to simulate ARGs under complex demographic scenarios involving geography and
selection that were previously infeasible, providing essential ground truth for
method evaluation. Simulation capabilities have continued to expand,
culminating in whole-genome ARG simulations for nearly 1.5 million individuals
based on a large human pedigree~\cite{andersontrocme2023genes}. 
A growing ecosystem of simulation tools now builds directly on tskit
(Table 1).


% Inference methods interoperate through tskit, enabling evaluation and reuse
% without imposing design choices.

The lack of scalable inference methods has been a major obstacle 
to practical application of ARG inference methods. Although there 
are many inference methods (see Wong et al.\cite{wong2024general}\ for a review),
tsinfer was the first to scale to hundreds of thousands of samples,
directly leveraging the tree sequence data
representation\cite{kelleher2019inferring}.
Many recent ARG inference methods have chosen to support tskit 
as an output format, in addition to their
own native representations (Table 1). 
This shared output layer enables inferred ARGs to
interoperate directly with simulators, facilitating systematic evaluation and
benchmarking against known ground truth. It also shifts the burden of format
conversion away from downstream users, who can instead rely on inference tools
to emit results in a common, well-defined representation. The scalability and
flexibility of this approach are illustrated by the recent inference of an ARG
for 2.48 million SARS-CoV-2 whole genomes, which occupies 32 MiB of storage and
can be loaded into memory in under a second\cite{zhan2025pandemic}.


% Shared representation unlocks fast, correct, and reusable downstream
% analysis.

Efficient storage and analysis of large genealogical datasets is a central
design goal of tskit, and the tree sequence representation enables substantial
performance gains in downstream analyses. 
For example, single-site 
population genetic statistics can be computed orders of magnitude faster than from
genotype matrices while using far less memory by operating 
on the underlying genealogical structure\cite{ralph2020efficiently}.
Tskit exposes a large API spanning statistical calculations 
and visualisation, with a
performance-critical core implemented in C and bindings available in Python,
Rust, and R. Its vectorised, table-first design allows zero-copy access to
underlying arrays, supporting high-performance analysis pipelines. 
As a result,
downstream tools inherit performance and correctness properties from a shared,
well-tested core.

% tskit 1.0 formalises this ecosystem as stable, long-term scientific
% infrastructure.

Tskit is now established as foundational infrastructure for population-scale
genealogical analysis. Table 1 summarises 56 published software tools that rely
on tskit, spanning simulation, inference, statistical genetics, visualisation,
and data processing. The release of tskit 1.0 marks the maturity of this
ecosystem and introduces explicit guarantees of long-term stability for the
data model and APIs. These guarantees, together with a global community of
contributors, allow tskit to be adopted with confidence as a durable scientific
resource and a common interchange format. As the field continues to develop,
challenges such as incorporating structural variation will require extensions
to the data model, but these will be pursued in a way that preserves stability
and interoperability. In this context, tskit 1.0 represents a transition from a
successful software library to stable infrastructure supporting
population-scale analyses of ancestry and recombination.


\begin{table}
\caption{The tskit software ecosystem. * data not available on OpenAlex}
\begin{scriptsize}
\begin{center}
\begin{tabular}{lllrr}
\toprule
name & interface & package & year & citations \\
\midrule
\multicolumn{5}{c}{\textsc{Simulation}} \\
\midrule
pyslim & Python & pypi, conda & 2025 & 1 \\
TwisstNTern & Python &  & 2025 & 0 \\
tstrait & Python & pypi, conda & 2024 & 8 \\
slendr & Python & CRAN & 2023 & 15 \\
cnasim & Python & pypi & 2023 & 10 \\
gridCoal & Python &  & 2022 & 5 \\
Geonomics & Python & pypi & 2021 & 20 \\
stdpopsim & Python & pypi, conda & 2020 & 233 \\
RADinitio & Python & pypi & 2020 & 49 \\
fwdpy11 & C, Python & pypi, conda & 2019 & 74 \\
SLiM & C &  & 2018 & 860 \\
msprime & C, Python & pypi, conda & 2016 & 806 \\

\midrule
\multicolumn{5}{c}{\textsc{ARG Inference}} \\
\midrule
SINGER & Python &  & 2025 & 12 \\
sticcs & Python &  & 2025 & 0 \\
POLEGON & Python &  & 2025 & 0 \\
Threads & Python & pypi & 2024 & 22 \\
espalier & Python & pypi & 2023 & 14 \\
sc2ts & Python & pypi & 2023 & 14 \\
tsdate & Python & pypi, conda & 2022 & 187 \\
ARGinfer & Python & pypi & 2022 & 37 \\
Relate & C &  & 2019 & 562 \\
tsinfer & C, Python & pypi, conda & 2019 & 373 \\
ARGneedle & Python & pypi & * &  \\

\midrule
\multicolumn{5}{c}{\textsc{Population Genetic Inference}} \\
\midrule
gaia & C &  & 2025 & 8 \\
mrpast & Python & pypi & 2025 & 0 \\
phlash & Python & pypi & 2025 & 0 \\
sparg & Python & pypi & 2025 & 0 \\
spacetrees & Python &  & 2024 & 18 \\
mapNN & Python &  & 2024 & 4 \\
gIMble & Python & conda & 2023 & 33 \\
disperseNN2 & Python & pypi & 2023 & 10 \\
dinf & Python & pypi & 2023 & 8 \\
GADMA & Python & pypi & 2022 & 23 \\
dnadna & Python & pypi, conda & 2022 & 8 \\
DISMaL & Python & pypi & 2021 & 7 \\
ReLERNN & Python & pypi & 2020 & 166 \\
momi2 & Python & pypi & 2019 & 170 \\

\midrule
\multicolumn{5}{c}{\textsc{Statistical Genetic Inference}} \\
\midrule
as-eGRM & Python &  & 2025 & 1 \\
arg-lmm & Python & pypi & 2025 & 0 \\
tslmm & Python &  &  &  \\
lgdm & Python & pypi & 2023 & 38 \\
egrm & Python & pip & 2022 & 44 \\

\midrule
\multicolumn{5}{c}{\textsc{Visualisation}} \\
\midrule
tsbrowse & Python & pypi & 2025 & 0 \\
ARGscape & Python & pypi & 2025 & 0 \\
tskit-arg-visualizer & Python & pypi & 2024 & 0 \\

\midrule
\multicolumn{5}{c}{\textsc{Data processing}} \\
\midrule
bio2zarr & python & pypi, conda & 2025 & 0 \\

\midrule
\multicolumn{5}{c}{\textsc{Analysis}} \\
\midrule
tscompare & Python & pypi & 2025 & 1 \\
twisst2 & Python &  & 2025 & 0 \\
tspop & Python & pypi & 2023 & 12 \\
\bottomrule
\end{tabular}
\end{center}
\end{scriptsize}
\end{table}


% Citations:

% - wong2024general
% - brandt2024promise
% - lewanski2024era
% - nielsen2024inference
% - kelleher2016efficient
% - kelleher2019inferring
% - ralph2020efficiently
% - andersontrocme2023genes
% - zhan2025pandemic


\section*{Acknowledgments}

\bibliography{paper}


\end{document}
