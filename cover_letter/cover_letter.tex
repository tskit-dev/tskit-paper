\documentclass{letter}


\usepackage{letterbib}
\usepackage[super,comma,sort&compress]{natbib}
% \usepackage[superscript,biblabel]{cite}
% \bibliographystyle{naturemag}

\signature{Jerome Kelleher}

\address{Big Data Institute\\University of Oxford\\UK}
\begin{document}

\begin{letter}{Nature Methods}

\opening{Dear Editors,}

I am writing on behalf of my coauthors to submit our manuscript entitled
\emph{Population-scale Ancestral Recombination
Graphs with tskit 1.0}
for consideration for publication in Nature Methods
as a Correspondence article.
The purpose of the manuscript is to announce the release of tskit 1.0,
marking an important milestone of maturity for the software.

The evolution of tskit over the past decade from
an aspect of standalone simulation program to a core piece of
software infrastructure for the field is complex, and an
understanding of this history is necessary to explain our
logic behind the format and venue for the present manuscript.
I hope that you will forgive this digression, and the bibliography
which we include because many of the key papers in this evolution
are not cited in the manuscript due to space restrictions.
Tskit began as part of the msprime coalescent
simulator\cite{kelleher2016efficient,baumdicker2022efficient},
which has become a standard and highly cited tool.
The core ARG functionality was subsequently extracted
from msprime to be become tskit and described in a paper
that defined core methology behind forwards-time simulation
of ARGs~\cite{kelleher2018efficient}.
The research program behind tskit has produced a series of
efficient algorithms described in publications
and implemented in tskit\cite{kelleher2019inferring,
ralph2020efficiently,tsambos2023link,fritze2024forest,
lehmann2025on}. Thus, while there are many publications
that discuss \emph{improvements to} tskit, there is currently
no publication that is specifically \emph{about} tskit.

Our motivation for the present manuscript is therefore
to provide this single authoritative overview of the high-level
goals of tskit, and to an appropriate resource for downstream
users to cite. The concise Correspondence format is ideal
for this, as it enforces brevity and clarifies the core
announcement of a long-term committement to stability,
are required for key infrastructure. We believe that this
will be of interest to a general Methods readership, as
scientific infrastructure enabling a rapidly growing field.


\closing{Sincerely,}

\nocite{*}
\bibliographystyle{plainnat}
\bibliography{references}

\end{letter}
\end{document}
