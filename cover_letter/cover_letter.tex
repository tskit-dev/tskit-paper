\documentclass{letter}

% \usepackage[superscript,biblabel]{cite}
% \bibliographystyle{naturemag}

\signature{Jerome Kelleher}

\address{Big Data Institute\\University of Oxford\\UK}
\begin{document}

\begin{letter}{Nature Methods}

\opening{Dear Editors,}

I am writing on behalf of my coauthors to submit our manuscript
\textbf{Population-scale Ancestral Recombination Graphs with tskit 1.0}
for consideration as a \textbf{Correspondence} in Nature Methods.
This article announces the release of tskit 1.0, marking a stable,
versioned data model and API for representing and manipulating ancestral recombination
graphs (ARGs) at population scale.

ARG-based approaches are increasingly used across population and statistical
genetics, but their practical adoption has been limited by fragmented data
representations and bespoke tooling.
Tskit provides a semantically complete and efficient representation of
genetic ancestry that enables interoperability across methods and scales
from thousands to millions of samples.
Through providing core infrastructure for a broad range of downstream software
(including widely used tools such as msprime and SLiM), as well as for many
bespoke analyses in individual publications, tskit underpins many current
ARG-based workflows.
The purpose of the
Correspondence is to document the transition of tskit to long-term stable
infrastructure, and to provide a concise, unified description of its design
principles and long-term stability guarantees. The explicit articulation of
these principles will allow downstream developers to build new methods on a
shared and dependable foundation with confidence, while aiding the
reproducibility of bespoke analyses. 
We intend this Correspondence to provide a stable, citable reference for the
scope, guarantees, and intended use of tskit, supporting reproducible and
interoperable workflows as methods and applications develop.

\closing{Sincerely,}

\end{letter}
\end{document}
