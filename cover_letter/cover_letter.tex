\documentclass{letter}


\usepackage{letterbib}
\usepackage[super,comma,sort&compress]{natbib}
% \usepackage[superscript,biblabel]{cite}
% \bibliographystyle{naturemag}

\signature{Jerome Kelleher}

\address{Big Data Institute\\University of Oxford\\UK}
\begin{document}

\begin{letter}{Nature Methods}

\opening{Dear Editors,}

I am writing on behalf of my coauthors to submit our manuscript entitled
\emph{Population-scale Ancestral Recombination
Graphs with tskit 1.0}
for consideration for publication in Nature Methods
as a Correspondence article.
The purpose of the manuscript is to announce the release of tskit 1.0,
marking an important milestone of maturity for the software.

We believe that this will be of strong interest to the readership
of Nature Methods for several reasons. Because of the limited
space in the article and number of citations allowed we will
cite more things here.

Firstly, as outlined in the
manuscript, there is now great interest in the practical application
of Ancestral Recombination Graphs across varied applications
in population and statistical genetics. Beyond the three review
articles articles published in 2024 that are cited in the manuscript,
there has been several other opinion pieces published on the
great promise of
ARGs\cite{harris2019database,tang2019genealogy,harris2023using}
and major results being achieved at increasing
pace~\cite{grundler2025geographic,speidel2025high}.

Secondly, tskit is not a collection of standard algorithms
but the result of a decade of active research. The efficient
algorithms underpinning tskit for
forwards-time simulation\cite{kelleher2018efficient},
population genetics
statistics\cite{kelleher2016efficient,ralph2020efficiently},
comparison metrics\cite{fritze2024forest},
local ancestry inference\cite{fritze2024forest},
and statistical genetics statistics\cite{lehmann2025on}.
Each of the publications has described some functionality
with tskit, showing state of the art performance
in all cases.

Tskit's leadership in the field is perhaps best illustrated
by the range of publications benchmarking specialised methods
against some subset of tskit's
functionality.\cite{dehaas2025enabling,walia2026compressive}.


Taken together, along with the large ecosystem of downstream
tools shown in the supplementary information shows that tskit is a leading piece
of infrastructure and a key part of a quickly growing field
of research. We are confident that tskit will continue to be
at the centre of this important field.

\closing{Sincerely,}

\bibliographystyle{plainnat}
\bibliography{references}

\end{letter}
\end{document}
